\section{Diversity of Dwarf Diffuseness}

% TODO: graph for these results!!!
There are only so many conclusions we can draw about the diversity of dwarf sizes with the given data. The data from \cite{mcconnachieOBSERVEDPROPERTIESDWARF2012} spans the range of $10^2$ to $10^9$ solar masses, with many of the galaxies observed being smaller than $3 \cdot 10^6$. The FIREbox data, however, only covers galaxies greater than $3 \cdot 10^6$. For this reason and others, the lines of best fit defined by $\ln(a)$ and $\gamma$ (see equation \ref{equ:linear-rel}) for each dataset do not perfectly line up. %TODO: \ref{fig}
The diffuseness parameter $\beta$ is defined relative to the line of best fit for a dataset. It is therefore important to note that the $\beta$ for individual galaxies across datasets have no meaningful comparison. However, we may still compare the overall diversity of the diffuseness. For FIREbox we have $\sigma_\beta = 0.28$, and for \cite{mcconnachieOBSERVEDPROPERTIESDWARF2012} we have a diversity of $\sigma_\beta = 0.66$, over twice as large. The simulated galaxies in FIREbox therefore follow a much stricter size-mass relationship than is expected from real-world observation.

This discrepancy is an example of the diversity of dwarf sizes tension, and \cite{salesBaryonicSolutionsChallenges2022} has already shown that it applies to many higher resolution simulations. FIREbox has a larger volume and contains more galaxies than those simulations, and is the first of its size to be detailed enough to resolve low-mass galaxies. It is therefore an important finding that the tension is not resolved in FIREbox, because that is evidence that the tension is \emph{not} caused by a constraint in simulation volume.


\section{Proximity to Host Galaxy}

We find that there is no correlation between a satellite's distance to its host galaxy and its diffuseness. 

\begin{figure}
    \centering
    \includegraphics*[width=\textwidth*2/3]{figs/me/beta-d.pdf}
    \caption{
        Shows the diffuseness of satellites as a function of distance to their host galaxies. There is no correlation between these parameters.
    }
    \label{fig:beta-d}
\end{figure}


\section{Proximity to Nearest Neighbor}

