% TODO: Summary of paper and results

\section{Size-mass-proximity relation}

We discovered that $\beta$ is correlation with the log of a satellite's proximity to its host galaxy. This relationship, as we will demonstrate, implies that we can include this distance as a new term in the formula for the size-mass relationship (equation \ref{equ:linear-rel}). To prove this, we will reverse engineer that equation starting from equation \ref{equ:beta}. We will begin by expressing $\beta$ in terms of $d_{min} / R_{vir}$ using the experimental results found in figure \ref{fig:beta-dmin-fit}.

\begin{equation}
    \beta = -0.18 \ln\left( \frac{d_{min}}{R_{vir}} \right)
\end{equation}

Where $-0.18$ is the slope of the line of best fit in this relationship. We can plug this into equation \ref{equ:log-log-with-beta}, giving us

\begin{equation}
    \ln \left(
        \frac{R_{50}}{1 kpc}
    \right)
    =
    \ln(a)
    + \gamma \ln \left(
        \frac{M_*}{M_\odot}
    \right)
    -0.18 \ln\left( \frac{d_{min}}{R_{vir}} \right)
\end{equation}

We take the exponent to reveal the final power law relationship.

\begin{equation}
    R_{50} = a
    \left( \frac{M_*}{M_\odot} \right)^\gamma
    \left( \frac{d_{min}}{R_{vir}} \right)^{-0.18}
    \text{kpc}
\end{equation}

And finally, we may plug in the experimental values of $a$ and $\gamma$ (see figure \ref{fig:size-mass}). This gives us a final equation of 

\begin{equation}
    R_{50} = 0.41
    \left( \frac{M_*}{M_\odot} \right)^{0.15}
    \left( \frac{d_{min}}{R_{vir}} \right)^{-0.18}
    \text{kpc}
\end{equation}

As we can see, a satellite's distance to its host has just as much of an influence on its radius as its mass does. 


\section{Further Work}