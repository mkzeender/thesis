
Understanding the composition and structure of galaxies and the role that dark matter plays in their organization is one of the most pressing topics in modern intergalactic physics. A common method to explore these questions is using simulations. A simulation allows us to choose plausible initial conditions and plausible laws of physics and test how the universe would behave under those conditions. We can then compare those results to experimental data to examine the accuracy of those initial assumptions. For instance, if we wanted to test Newton's theory of gravity in our solar system, we could run a numerical simulation of Newton's equation, starting from a past known position of the planets, and test whether or not the simulated motion of the planets aligns with our real astronomical observations. Likewise, we can test our theories about dark matter and gravity by running galaxy simulations.


Our current leading theory for dark matter's role in galaxy evolution is the Dark Energy and Cold Dark Matter (\lcdm) model (\cite{salesBaryonicSolutionsChallenges2022}). This theory provides a framework for physical simulation that incorporates a non-interacting ("cold") model for dark matter and the cosmological constant model of dark energy. Dark matter is assumed to not interact with either itself or "normal" baryonic matter, except through gravity. Such a model leads to large dark matter halos around galaxies that do not collapse into disks, which is consistent with observational data (\cite{feldmannFIREboxSimulatingGalaxies2022}). Since not much else is known about dark matter, and we have yet to find a non-gravitational interaction between dark matter and baryonic matter, this assumption is widely accepted in practice. Likewise, \cite{feldmannFIREboxSimulatingGalaxies2022} assumes dark energy to be the cosmological constant $\Lambda$, which is a degree of freedom in in the Einstein Equation that adds a net offset to the energy density of a vacuum. While the cosmological constant model of dark energy is quite popular and consistent in most ways with observational data, it is not the only model of dark energy. \cite{bassiCosmologicalEvolutionBimetric2023} pose an alternative: the Bimetric gravity model. This theory hypothesizes that the graviton---the theoretical particle that causes gravitational interactions---has mass. \cite{bassiCosmologicalEvolutionBimetric2023} show that a graviton with a non-zero mass could cause a net pressure in the vacuum, negating the need for the cosmological constant. They argue that the Bimetric theory of gravity could also resolve the Hubble Tension. This is an inconsistency between the measurement of the Hubble Constant at large and small scales (\cite{senCosmologicalObservationsAllow2022}), a calculation that relies on the cosmological constant being just that: a constant. If more evidence can be found in support of it, the Bimetric model may replace the \lcdm\space model, but for now the latter is still the most widely used.

When creating a galaxy simulation, physicists must also incorporate baryonic processes, the physics of ordinary matter. Baryonic properties incorporated into galaxy simulations may include gas density, pressure, temperature, star formation rate,

% TODO: talk about the history of baryonic processes. Compare and contrast

It is a common myth that if we can simulate something, we must be able to fully understand it. Unfortunately, this is not generally true. The galaxy simulations we use are so complex and detailed that it is often very difficult to determine what physical assumptions or initial conditions cause certain behaviors. Instead, we must analyze the simulation data using similar techniques we use to analyze observational data (although it is much easier to collect data about a simulation than about a real system of galaxies). From there, we compare the results to our expected results from experimental data to test our theories.

As one would expect, the simulation data does not always line up with the observational data; instead, there are a number of known tensions between the two. One such tension is the size-mass relation of dwarf galaxies. Observational data of dwarf galaxies near the Milky Way suggests that the correlation between the mass and size of satellite dwarf galaxies is not as strong as simulations seem to predict (\cite{salesBaryonicSolutionsChallenges2022}). 


\section{FIREbox Galaxy Simulation}
The FIREbox (\cite{feldmannFIREboxSimulatingGalaxies2022}) simulation is the most in-depth galaxy simulation ever performed as of the date of this thesis. It does not have the largest volume, nor is it the most detailed; sub-simulations such as FIRE in the Field (\cite{fittsFireFieldSimulating2017}) zoom in closer, to a particle size as low as 500 solar masses. FIREbox, however, has the total combined resolution and incorporates a balance of detail and scale.

Scale and resolution are an important component of galaxy simulations. Previous iterations of galaxy simulation needed to choose between larger volume and higher resolution. The large volume simulations allow scientists to closely study the interactions between galaxies and systems of galaxies, and to collect large amounts of statistical information about these galaxies (\cite{feldmannFIREboxSimulatingGalaxies2022}). However, the large resolution sacrificed physics accuracy and therefore realism; a higher resolution "zoom in" simulation allows us to better simulate the internal physics of the galaxies themselves (\cite{feldmannFIREboxSimulatingGalaxies2022})



\section{Oh}

\section{Another section}

This second section is, obviously, 1.2.

\subsection{A subsection}

\subsubsection{A subsubsection}

Subsubsections are still smaller sections.  By default, this is the finest subdivision of a chapter in \LaTeX, and they will \emph{not} appear in the table of contents.  

\subsection{A useful command}
...

\section{Some figures}

You will surely want to add figures to your thesis to help explain your ideas.  There are a number of different ways to include such things, but the most typical way would be to generate the figure in another piece of software (\texttt{MATLAB, Mathematica, Adobe Illustrator, \ldots} and simply include it in your \LaTeX ~code.  This will require use of the \emph{figure} environment.\footnote{there are many other possible environments to include figures, such as wrapfigure, but these will require including additional packages \ldots}  See this document's \LaTeX ~code for details . . .

% Here is the figure environment.  The [] after \begin{figure} are an optional argument that tells LaTeX where to try to place the figure.  If you do not include it then it will figure out the `best' place to put things on its own, but on occasion the choices it makes are a bit strange.  Even so, you should probably try to let it make the decisions whenever possible, and only come back to enforce new locations after you have essentially finished your document.  Otherwise you may find that your instructions are no longer appropriate if you add text elsewhere in the document that changes how things are alligned.  Anyhow, [h] instructs LaTeX to put th e figure `here', and adding an exclamation point [h!] means REALLY, PUT IT HERE.  [t] and [b] mean to put the figure at the top or bottom of a page, respectively.
\begin{figure}[ht!]

%\centering centers the figure on the page.  This is convention.
\centering

% The \includegraphics command is the default way to include a figure.  The first optional argument in [] brackets defines how large the figure should be on the page.  This can be defined in inches, centimeters, or as a fraction of the \textwidth.  Then, in the curly {} braces you will put the file name for the figure.  Here it has been stored in a subfolder entitled Figures.  You are encouraged to give your figures descriptive file names, as this will help future students to figure out what figure corresponds to which file without having to read your LaTeX code.  

%\caption will create a figure caption.  Putting an asterisk after it (as in \caption* ) will cause the caption to not be numbered and to not appear in the list of figures.  The [] argument is optional but recommended; this is the short-form caption which will appear in the list of figures.  The {} argument is required, and gives the full caption which will appear in the main text.  The first few words of this caption will be used in the list of figures if no [] form is given.  
\caption[Short-form caption]{Long-form caption that appears in main body of the document}

% the \label gives a reference for this figure so that you can refer to it in the text by its number in a way which will be automatically updated if you change the document.  THIS IS A GOOD IDEA; having automatically updated references for figures, equations, etc, will keep your document in order even as you continue to update it over a period of months.  This reference can be called in the text using the \ref tag.
\label{fig:aFigure}
\end{figure}



% Here is the same figure, but now using the wrapfigure environment.  This allows you to wrap your text around the figure itself.  The optional [] argument specifies how many lines of text should be wrapped.  The first {} argument specifies where on the page the figure should live (here the right side is specified).  The second {} argument defines how much real estate on the page will be allocated for the figure block; here it is 45% of the page width.
\begin{wrapfigure}[11]{r}{0.45\textwidth}
\centering
% \vspace creates vertical white space in the document.  Here we are creating 0.1 lines of text worth of negative white space, effectively moving our graphic up in the document by one line.  I find this lines things up better within the text in this case, but you may need to manipulate this to make the alignments look correct.
\vspace{-.1\baselineskip}
\caption[Another short-form caption]{A figure included using the wrapfig environment}
\label{fig:anotherFigure}
\end{wrapfigure}

Here I have added a table, because tables are also useful. This table has nothing to do with the rest of the material in this thesis template, but you should probably only add relevant tables.
\begin{table}[tbh]
\begin{center}
%\caption[]{\em{Here we show the continuum sensitivity required per band.}}
\begin{tabular}{ccccccc}
\hline \noalign {\smallskip}
Name & SpT & Dist. & Age & 3$\sigma$ M$_{\rm dust}$ & 3$\sigma$ CO(3-2) limit & Disk indicator \\
 & & (pc) & (Myr) & limit (M$_{\oplus}$) &  (mJy km s$^{-1}$)\\
\hline \noalign {\smallskip}
J0226 & L0 & 46.5 & 45 & 0.01 & 24 & Pa$\beta$, IR\\
J0501 & M4.5 & 47.8 & 42 & 0.01 & 23 & H$\alpha$, IR\\
J1546 & M5 & 59.2 & 55 & 0.01 & 14 & HeI, [OI], H$\alpha$, IR\\
J0446 A/B & M6/M6 & 82.6/82.2 & 42 &  0.027 & 18 & H$\alpha$, IR\\
J0949 A/B & M4/M5 & 79.2/78.1 & 45 &  0.024 & 17 & H$\alpha$, IR\\
LDS 5606 A/B & M5/M5 & 84/84 & 30-44 & 0.027 & 19 & H$\alpha$, IR, UV\\
\hline \noalign {\smallskip}
\end{tabular}
\end{center}
\end{table}


