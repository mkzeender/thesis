
Understanding the composition and structure of galaxies, and the role that dark matter plays in their organization, is a pressing topic in modern cosmology. A common method to explore these questions is with numerical simulations. This approach allows us to choose plausible initial conditions and laws of physics in order to test how the universe would behave. We can then compare those results to real-world observations to determine the accuracy of those physical assumptions, and we can predict new effects that can be confirmed by observations. For instance, if we wanted to test Newton's theory of gravity in our solar system, we could run a numerical simulation of Newton's equation using a known initial position of the planets, and test whether the simulated motion of the planets aligns with reality. Likewise, we can test our theories about dark matter and gravity by running cosmological simulations. These simulate a space full of stars, gas, and dark matter that evolves into a system of galaxies.

\section{Physical models}

Our current leading theory for dark matter's role in galaxy evolution is the cosmological constant plus cold dark matter (\lcdm) model \citep{whiteCoreCondensationHeavy1978}. This theory provides a framework for cosmological simulations that incorporates a non-interacting model for dark matter and the cosmological constant model of dark energy. The "cold" dark matter model means that we assume that dark matter particles interact with neither each other nor "normal" baryonic matter, except through gravity. This is in contrast to other competing theories such as superfluid dark matter, which interacts with itself to form a superfluid \citep{delucaSuperfluidDarkMatter2023}. These dark matter models predict large dark matter halos around galaxies \citep{feldmannFIREboxSimulatingGalaxies2022}. Such predictions are consistent with our observations of gravitational lensing---visual distortions due to gravity. They also are a better (though imperfect) fit for the rotation curves of galaxies---the orbital velocity of stars as a function of their distance to their galaxy's center \cite{salesBaryonicSolutionsChallenges2022}. Not much is known about dark matter beyond its gravitational effects on baryonic matter, and we have yet to discover a non-gravitational interaction between these two.

\begin{figure}
    \centering
    \includegraphics*[width=\textwidth*2/3]{figs/gravity_lensing.jpg}
    
    \caption{From: nasa.gov. A real-world example of gravitational lensing. In the center is an elliptical galaxy (yellow). The blue ring, known as an Einstein Ring, is actually another galaxy that lies behind the first one. It is visually distorted into a ring shape by the gravitational lensing of the yellow galaxy. Without dark matter (or a competing theory), the lensing effect would be much weaker.}
    \label{fig:grav-lensing}
\end{figure}

The \lcdm\* paradigm assumes dark energy to be the cosmological constant $\Lambda$, which is a degree of freedom in in the Einstein Equation that adds a net offset to the energy density of a vacuum. However, there are alternative theories of dark energy; \cite{bassiCosmologicalEvolutionBimetric2023} shows that the Bimetric gravity model could also explain the effects of dark energy. If more evidence can be found in support of these alternative models, then Bimetric and/or superfluid dark matter may replace \lcdm\space as the leading cosmological model.

When creating a cosmological simulation, astrophysicists must also incorporate baryonic processes, the physics of ordinary matter. These processes include chemistry, thermal physics, and the formation, evolution, and feedback of stars. Our current computers limit us such that we cannot resolve individual stars within galaxies \citep{feldmannFIREboxSimulatingGalaxies2022} because they are simply too small, and stellar physics at these scales is not fully understood. Earlier simulations, such as \cite{bournaudISMPropertiesHydrodynamic2010}, were forced to ignore stellar processes entirely in favor of gaseous ones. They found that the simulated galaxies grew too massive and cooled too quickly compared to real galaxies. This tension was resolved by the creation of the "Feedback in Realistic Environments" (FIRE) physics model \citep{hopkinsFIRE2SimulationsPhysics2018}. FIRE adopts a subgrid model that simulates large chunks of matter (referred to as particles), each containing many stars. It then uses the estimated number of stars in each particle to simulate stellar feedback. 

\section{Low-mass galaxies and their tensions}

As telescopes have improved, there has been an increase in observations of low-mass "dwarf" galaxies. Before this, the \lcdm\* model was questioned because it predicted the existence of many more low-mass galaxies than had been observed in the region around the Milky Way \citep{klypinWhereAreMissing1999}. According to \cite{salesBaryonicSolutionsChallenges2022}, enough low-mass galaxies have been discovered in recent years to resolve this tension. However, the sudden influx of low-mass galaxy observations has provided astrophysicists number of new tensions. The diversity of the size-mass relation of low-mass galaxies is one such tension. Observational data of low-mass galaxies near the Milky Way suggests that the correlation between the mass and size of satellite low-mass galaxies is not as strong as simulations seem to predict \citep{salesBaryonicSolutionsChallenges2022}. 

\begin{figure}
    \centering
    \includegraphics*[width=\textwidth*2/3]{figs/sales/fig4.pdf}
    \label{fig:sales-size-mass}
    \caption{
        From: \cite{salesBaryonicSolutionsChallenges2022}. A comparison of the sizes and masses of low-mass galaxies from zoom-in simulations and reality. The $y$ axis plots $r_{50}$ and the $x$ axis plots $M_{50}$. The gray squares depict real galaxies from the Local Group, whose data was compiled by \cite{mcconnachieOBSERVEDPROPERTIESDWARF2012}. The other colors depict data from various simulations other than FIREbox \citep[refer to][]{salesBaryonicSolutionsChallenges2022}. As one can see, the dwarfs from the Local Group show much diversity in their size-mass ratios. The simulated galaxies, however, show more consistency, especially when compared to others from their own simulation (for example, notice that the orange circles are all clustered together). This paper will extend this analysis to the FIREbox data.
    }
\end{figure}


\section{Tension in the size-mass relation}

The observed low-mass galaxies near the Milky Way have widely varying radii compared to their masses. In other words, both diffuse and compact low-mass galaxies are relatively common. However, zoom-in galaxy simulations, including \cite{fittsFireFieldSimulating2017}, tend to form galaxies with much stricter size-mass ratios \citep{salesBaryonicSolutionsChallenges2022}. Some may argue that these discrepancies are caused by numerical inaccuracy. However, even the simulations with the highest numerical resolution such as \cite{wheelerBeItTherefore2019} lack diversity in size-mass ratios for low-mass galaxies. 

The diversity of sizes of low-mass galaxies must therefore be caused by something else. Tidal disruption could be the answer. When a low-mass galaxy passes by a larger galaxy, its dark matter halo can be destroyed by the tidal force exerted on it, according to \cite{morenoGalaxiesLackingDark2022}. This can also lead to the creation of an ultra-compact low-mass \citep{applebaumUltrafaintDwarfsMilky2021}, because the galaxy will lose its outer regions. It is therefore plausible that close interactions between galaxies is what causes variance in size.

