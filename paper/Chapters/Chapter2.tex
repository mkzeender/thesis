% Another mock chapter

\section{FIREbox Galaxy and Halo catalog}
Over the course of the simulated universe's evolution, 1201 snapshots of the state of the universe were collected (\cite{feldmannFIREboxSimulatingGalaxies2022}). They were approximately evenly spaced out in time and included the positions of the particles, their densities, metalicities, star-formation rates, and other properties. The particle data was reduced by grouping the particles into their respective galaxies and dark matter halos. They used the AMIGA Halo Finder (AHF; \cite{knollmannAhfAMIGAHALO2009} to sort the halos into categories of halos and sub-halos, which in turn allowed them to categorize the galaxies by host and satellite galaxies respectively. The reduced data, known as the galaxy and halo catalog includes galaxy information such as CM position, radius (measured using 

AAAAAAAAAAAAAAAAAAAAAAAAAAAAAAAA

\section{Host Proximity}

\section{Mass-Size Deviation}







AAAAAAAAAAAAAAAAAAAAAAAAAAAAAAAAAAAAAAAAAAAAAAAa
Here is a second mock chapter.  As far as the \LaTeX ~is concerned, it is in no way different from the introduction excepting that it appears after it in the main .tex file.  As before, it can be populated with sections, subsections, figures, etc. as you see fit.

In fact, you will probably write perhaps three to six chapters for your thesis depending on how your work is most effectively organized.  Most theses will contain an introduction, at least one `body' chapter, and some sort of conclusions/future directions chapter.  Most theses will also include an appendix or two \ldots