% Another mock chapter

\section{FIREbox Galaxy and Halo catalog}
Over the course of the simulated universe's evolution, 1201 snapshots of the state of the universe were collected (\cite{feldmannFIREboxSimulatingGalaxies2022}). They were approximately evenly spaced out in time and included the positions of the particles, their densities, metalicities, star-formation rates, and other properties. The particle data was reduced by grouping the particles into their respective galaxies and dark matter halos. They used the AMIGA Halo Finder (AHF; \cite{knollmannAhfAMIGAHALO2009}) to sort the halos into categories of halos and sub-halos, which in turn allowed them to categorize the galaxies by host and satellite galaxies respectively. The reduced data, known as the galaxy and halo catalog includes galaxy information such as CM position, radius (measured using 

AAAAAAAAAAAAAAAAAAAAAAAAAAAAAAAA

\begin{figure}
    \includegraphics*[width=\textwidth]{figs/feldmann/fig1.pdf}
    \label{fig:feldmann-visual}
    \caption{
        From: \cite{feldmannFIREboxSimulatingGalaxies2022}. A representation of the FIREbox simulation. The first two rows depict the state of the simulation at three different time points; the rightmost images depict the simulation in the present time. The top row depicts dark matter in blue and stellar matter in white, while the middle row depicts gas. 
    }
\end{figure}

\begin{figure}
    \includegraphics*[width=\textwidth]{figs/sales/fig4.pdf}
    \label{fig:sales-size-mass}
    \caption{
        From: \cite{salesBaryonicSolutionsChallenges2022}. A comparison of the sizes and masses of dwarf galaxies from simulations and reality. The gray squares depict the half radius of 
    }
\end{figure}


\section{Host Proximity}

\section{Mass-Size Deviation}
