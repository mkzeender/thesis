% Here we have your executive summary.

Cosmological simulations are an important part of our understanding of the universe. They allow us to test our theories about cosmology by seeing if the simulated behavior created by those theories aligns with the real world. The FIREbox galaxy simulation uses the theory of \textbf{\lcdm}, the current most popular model of dark matter and dark energy. FIREbox models a universe of over 1700 galaxies from their initial creation all the way through the present day. This present state can be compared to real-world galaxies to determine what simulated characteristics align with the real ones.

When a certain characteristic of the simulation does not line up with observational data, it is called a \textbf{tension}. One such tension is that of the diversity of low-mass galaxy sizes. The low-mass galaxies near the Milky Way (known as the \textbf{Local Group}) have much variation in radius compared to their mass. In other words, they range from very compact to very fluffy. However, most historical \emph{simulated} galaxies have a much stricter size-mass ratio. This discrepancy is the driving motivation for this thesis.

To examine the tension further, I compare the fluffiness of galaxies from the FIREbox cosmological simulation with other galactic properties. I define the \textbf{fluffiness} ($\beta$) of a galaxy to be how much it deviates from the expected size-mass ratio. This parameter is useful for our study because, by design, it is independent of mass for low-mass galaxies. We are therefore free to correlate it with other parameters.

One theory for this tension is tidal disruption. Tidal disruption occurs when a galaxy comes in close contact with another, and its outer layers get stretched by the other's gravity. Tidal forces might therefore increase fluffiness. I tested this by examining the relationship between a satellite and its host galaxy (satellites are galaxies that orbit other galaxies). I compared the fluffiness of satellite galaxies with the distance to their hosts: both in the present and when they were at their \emph{minimum} distance. I found that when you express the distance as a fraction of the host galaxy's virial radius, there is a correlation. In other words, the closer a satellite gets to its host, the more fluffy it becomes.