% Here we have your executive summary.

Cosmological simulations are an important part of our understanding of the universe. They allow us to test our theories about cosmology by seeing if the simulated behavior created by those theories aligns with the real world. The FIREbox galaxy simulation uses the theory of \textbf{\lcdm}, the current most popular model of dark matter and dark energy. It models a box of over 1700 galaxies from their initial creation all the way through the present day. This present day state can be compared to real world galaxies to determine what simulated characteristics align with the real ones.

When a certain characteristic of the simulation does not line up with observational data, it is called a \textbf{tension}. One such tension is that of the diversity of low-mass galaxy sizes. The low-mass galaxies near the Milky Way (known as the \textbf{Local Group}) have much variation in radius compared to their mass. In other words, they range from very compact to very diffuse. However, most historical simulated galaxies have a much stricter size-mass ratio. This discrepancy is the driving motivation for this thesis.

To examine this tension further, I compare the diffuseness of galaxies in FIREbox with other galactic properties. I define the \textbf{diffuseness} ($\beta$) of a galaxy to be how much it deviates from the expected size-mass ratio. 